\documentclass{article}
\usepackage{listings}
\title{Rapport - Projet th\'eorie des graphes}
\author{ABDELMOUMENE Djahid\\\&\\ Hadjer}

\begin{document}
\maketitle
\section{Introduction}
			L'id\'ee principale de ce projet est de trouver le plus court chemin entre deux sommets d'un graphe orient\'e.\\
			Le graphe qui nous a \'et\'e fourni - la carte des pistes de serre chevalier - correspond \`a un graphe orient\'e muni d'une fonction de pond\'eration positive, 
			c'est \`a dire que tous les arcs ont un poids positif. Ce graphe n'est pas un graphe simple \`a premi\`ere vue (existence de plusieurs arcs 
			entre les m\^me sommets), mais on pourra voir dans les sections suivantes qu'on pourra le consid\'erer ainsi.

\section{Execution}
Pour executer le program, il suffit d'ouvrir le fichier 'index.html' avec le navigateur firefox.
	\section{Graphe}
			La premi\`ere \'etape de ce projet \'etait d'analyser la carte de la station de ski fournie et d'en tirer le graphe correspondant. 
			On a donc consid\'er\'e les pistes de ski et les remont\'ees m\'ecaniques comme des arcs, et les points d'intersection, de d\'epart et 
			d'arriv\'ee de ces pistes comme des sommets.
		\subsection{le graphe}
			le graphe est stock\'e comme un liste d'adjacence, donc le graphe est un tableau des objets des sommets. 
		\subsection{Sommets du graphe}
			\paragraph{Consid\'eration des sommets :\\}
			On consid\`ere comme sommets du graphe, les points de d\'epart et d'arriv\'ee des pistes et les points o\`u les pistes se croisent. 
			La carte fournie a \'et\'e simplif\'ee puisque les sommets correspondant \`a des points o\`u se croisent plusieurs pistes sont en fait des zones 
			(et non des points) o\`u se croisent ces pistes.\\
			Les sommets n'ont pas des noms, mais il sont d\'efini par leur indice dans le tableau de graph, et avec leur coordonn\'ees (cart\'esiens) dans l'image des pistes (2362x1013).
			\paragraph{Indices des sommets :\\}
			Chaque sommet a \'et\'e rep\'er\'e par un indice allant de 0 \`a 147 : \textbf{on a donc 148 sommets au total}.
			\paragraph{Coordonn\'ees des sommets :\\}
			Tout sommet \`a un coordonn\`ee cart\'esien dans l'image des pistes (2362x1013), et c'est repr\'esent\'e comme un tableau de taille 2 ou le premier element est le x et le deuxi\`eme est le y.
			\paragraph{Les voisins de sommet}
			Tout les sommet ont un objet (tableau de hachage) o\`u chaque voisin est represent\'e par son indice dans le tableau de graphe, et la valeur est l'objet de l'arc.
			Exemple d'un objet des voisins:\\
	\lstinputlisting[firstline=1260,lastline=1315,title=extrait de data.json]{./data/data.json}
		\subsection{Arcs du graphe}
			\paragraph{Consid\'eration des arcs :\\}
			Les arcs du graphe correspondent bien \'evidemment aux pistes de ski. On a alors deux types d'acrs : des \textit{descentes} et des
			\textit{remont\'ees}.\\
			Un simplification du graphe a \'et\'e effectu\'ee : s'il existe plusieurs arcs de m\^eme extr\'emit\'e de d\'epart et d'arriv\'ee, 
			on ne retient que l'arc de poids le plus faible. La raison est que pour calculer un plus court chemin, l'arc de poids faible est 
			le seul qui va etre pris en compte, peu importe le nombre des autres arcs liant ces mêmes sommets.
			\paragraph{Infomations stock\'ees dans les arcs :}\subparagraph{Le nom:}Chaque arc \`a un nom unique, vu que la majorit\'e des arcs ont des noms, si un arc n'as pas un nom il prend le nom de l'arc le plus proche mais avec un indice ajout\'e. eg: arc ,arc1, arc2..
			\subparagraph{Les coordonn\'ees}Tout les arcs ont un tableau des points dans l'image (2362x1013), et ces coordonn\'ees sont utilis\'ees pour dessiner les arcs, o\`u on connect chaque deux points cons\'ecutifs  avec un ligne.
			\subparagraph{La longueur de l'arc}
			Chaque arc a un longueur, qui est la somme des distance de chaque deux points cons\'ecutifs dans le tableau des coordonn\'ees de l'arcs.
			\subparagraph{Couleur et poids des arcs}
			les arcs ont un  qui represente le couleur ou le type de l'arc, 
			Les diff\'erents types d'arcs ont chacun leur propre couleur qui a \'et\'e rep\'er\'ee par un chaine des caractéres. voici la liste des couleurs et types possibles avec le poids:\\\\
			\textbf{Les pistes} (la couleur, poids un skieur debutant, poids pour un skieur expert):
			\begin{itemize}
			\item{"V": vert,  longueur*1.05/10, longueur*1/10}
			\item{"B": blue,  longueur*1.4/10 , longueur*1.1/10}
			\item{"R": rouge, longueur*2.2/10 , longueur*1.2/10}
			\item{"N": noir,  longueur*3/10   , longueur*1.3/10}\\\\
			\textbf{Les remont\'ees} (le type, le poids):
			\item{"TELECABINE": telecabine, longueur*0.5/10}
			\item{"OEUF": oeuf, longueur*0.75/10}
			\item{"TELESIEGE": telesi\`ege, longueur*0.9/10}
			\item{"TELESKI": teleski, longueur*1/10}
			\end{itemize}

			Voici la fonction qui permet la conversion du temps de parcours de l'arc et de sa couleur en poids:
	\lstinputlisting[lastline=23,,title=dijkstra.js]{./src/dijkstra.js}

\subsection{le fichier d'entr\'ee}
Le fichier d'entr\'ee est en format JSON, o\`u il existe les objets (tableau de hachage) represent\'e par des \{ \}, et des tableaux represent\'e par des [ ].


cette format nous permet de structurer l'objet de graphe dans le fichier de l'information.
\section{L'algorithme}
rien
\end{document}